\documentclass[小V的日麻笔记.tex]{subfiles}
\begin{document}
\chapter{做牌}
\section{基本手筋与手顺}
二阶堂亚树在红亚树说过,无论什么巡目,随意的切牌都不会有好下场。因此在做牌方面要学会从牌效上打牌。
那么\textbf{牌效}是什么?红亚树上定义的牌效是,打什么牌进张面最广。在鱼谷的书当中,定义的是平衡距离听牌的速度与和牌的速度,来做出最佳的选择。无论是哪种定义方式也离不开最基本的手筋与手顺。

那么什么是手筋,什么是手顺?
手筋,指的是基于战略做出的打牌方针
手顺,则是顾名思义切牌的顺序

所以来学习基础的手筋把。

一般一副手牌的切牌顺序如下:
\begin{itemize}
\item 在面子足够的情况下切有筋的端牌(既有4的时候切浮牌1,有6的时候切浮牌的9)

\textbf{例: 宝牌 \fmahjong{5z}}

\mahjong{14m1799p23467s554z}
\begin{itemize}
\arrowitem 切 \fmahjong{1m}
\arrowitem 能够组成面子的已经有\fmahjong{799p-234s-67s-55z}这4组
\arrowitem 即使切\fmahjong{4z},摸到了\fmahjong{23m}还是要切掉\fmahjong{1m}
\end{itemize}

\item 字牌
\begin{itemize}
\item 首先要考虑的是有无平和?因为平和需要非役牌的雀头

\textbf{例: 东1局\ 西家\ 第1巡\ 宝牌 \fmahjong{7p}}

\mahjong{23678m36789p34s25z}

\begin{itemize}
\arrowitem 这手牌能看出是个平和的,优先处理掉会减平和那一翻的\fmahjong{5z}会更佳
\end{itemize}

\item 然后就是最没价值的字牌

\textbf{例: 东1局\ 西家\ 第1巡\ 宝牌 \fmahjong{7p}}

\mahjong{13668m35799p3s125z}
\begin{itemize}
\arrowitem 多张字牌的时候,如果没有手役,一般都是客风→役牌→双风牌
\arrowitem 因此这手牌价值最小的就是客风的 \fmahjong{2z}
\arrowitem \fmahjong{1z}留最后是因为对自己来说只是单纯的役牌,然而优先打出去被庄家碰出有了确定的2翻的话是不利的,应该等人打出之后跟着打
\end{itemize}
\end{itemize}

\item 端牌

处理好字牌,筋上的端牌之后就是浮牌的端牌了

\textbf{例: 宝牌 \raisebox{-3mm}{\mahjong{5z}}}

\mahjong{1668m3799p2788s55z}

\begin{itemize}
\arrowitem 这一手牌只有1张价值最低的端牌,那就是 \raisebox{-3mm}{\mahjong{1m}}
\arrowitem 牌效率上讲数牌越往中间靠就越好。
\end{itemize}

但端牌之间也有一个比较,就是看看有没有相差4的的牌。

\textbf{例: 宝牌 \fmahjong{5z}}

\mahjong{15m167p234779s555z}

\begin{itemize}
\arrowitem 这手牌有两张浮牌,分别是 \fmahjong{1m} 和 \fmahjong{1p}
\arrowitem 但最佳留下的应该是 \fmahjong{1m}, 因为跟 \fmahjong{5m} 相差了4,换句话说一旦摸进了 \fmahjong{3m} 就能形成 \fmahjong{135m} 的两坎形式
\arrowitem 因此这题应该切 \fmahjong{1p}
\end{itemize}

\item 但是也有一些特殊情况,比如跟一色手,七对子,全带,对对等有手役的时候

\textbf{例: 宝牌 \fmahjong{7p}}

\mahjong{113467889m68s225z}

\begin{itemize}
\arrowitem 这手牌很明显的万子染手,这种情况下就应该从最中间也是较为危险的\fmahjong{6s}开始
\end{itemize}

\textbf{例: 宝牌 \fmahjong{9m}} 

\mahjong{14688m15p12s3345z-9m}

\begin{itemize}
\arrowitem 这一道题来自红亚树,选择的第一打是 \fmahjong{5p}
\arrowitem 手牌一般,看看能不能往混一色发展,需要注意的是此时他家打出\fmahjong{3z}的时候是不能碰的,至少要摸成了宝牌对子或者役牌对子才能决定混一色路线。
\arrowitem 况且这牌还有可能发展成一气通贯,总之副露的混一色本身也只有2翻,打点很低,至少要有额外的1翻或者2翻达到满贯的目的就最好了。
\end{itemize}

\end{itemize}

\section{4张牌的复合形手筋}
4张牌的复合形有三种,分别是像\fmahjong{3456m}的\emph{4连形}, \fmahjong{2334p}的\emph{中膨形}, 以及\fmahjong{1123s}的\emph{亚两面形},一般情况下但有浮牌跟复合形的牌在手牌的时候都会留下复合形


\textbf{例: 宝牌 \fmahjong{5z}} 

\mahjong{3678m3456p2379s55z}

\begin{itemize}
\arrowitem 切 \fmahjong{3m}
\arrowitem 因为\fmahjong{3456p}的4连形会有以下的改良 \newline
\fmahjong{23456p} \newline
\fmahjong{34567p} \newline
\fmahjong{34456p} \newline
\fmahjong{34556p}
\end{itemize}

\textbf{例: 宝牌 \fmahjong{5z}} 

\mahjong{2334m1237p345699s}

\begin{itemize}
\arrowitem 这题既有 \fmahjong{3456s}的4连形也有\fmahjong{2334m}的中膨形,无论那边的复合形都有不错的延伸改良
\arrowitem 浮牌的\fmahjong{7p}的好形靠张也只有\fmahjong{6p}和\fmahjong{8p}
\arrowitem 而中膨形则有以下的改良 \newline
\fmahjong{12334m} \newline
\fmahjong{22334m} \newline
\fmahjong{23344m} \newline
\fmahjong{23345m}
\item[\ding{213}] 因此切 \fmahjong{7p}
\end{itemize}

纵使听牌了,中膨形也要比单张浮牌有利

\textbf{例: 宝牌 \fmahjong{5z}} 

\mahjong{2334m1115p234789s}

\begin{itemize}
\arrowitem 假如听\fmahjong{5p},那么好形改良也只有以下3种 \newline
\fmahjong{11145p} 打 \fmahjong{1p} \newline
\fmahjong{11156p} 打 \fmahjong{1p} \newline
\fmahjong{11135p} 打 \fmahjong{5p} \newline

\arrowitem 但是中膨形的听牌改良就有4种,跟上一道例题一样
\arrowitem 因此切 \fmahjong{5p}
\end{itemize}

复合形之间呢也有优劣比较的,比如以下的几道例题
\begin{itemize}
\item 4连形之间的比较

\textbf{例: 宝牌 \fmahjong{5z}} 

\mahjong{5678m3456p2378s55z}

\begin{itemize}
\arrowitem 这题比较的是 \fmahjong{5678m} 跟 \fmahjong{3456p}的这两个4连形
\arrowitem 
\begin{tabular}{|c|c|}
\hline 
万子部分的改良 & 饼子部分的改良 \\
\hline 
\fmahjong{45678m} & \fmahjong{23456p} \\
\hline 
\fmahjong{56678m} & \fmahjong{34556p} \\
\hline 
\fmahjong{56678m} & \fmahjong{34456p} \\
\hline 
\fmahjong{56778m} & \fmahjong{34567p} \\
\hline 
\end{tabular}

\arrowitem 对比得出万子部分成两面的有3种,3面的而只有1种;而饼子部分成两面则有2种,成3面的有2种。因此饼子方面更胜一筹
\arrowitem 所以这牌切用途最小的\fmahjong{8m}
\end{itemize}

\item 中膨形之间的比较

\textbf{例: 宝牌 \fmahjong{5z}} 

\mahjong{233499m1223p2227s}

\begin{itemize}
\arrowitem 这题虽然有2个中膨形,然而用端牌组成的中膨形价值是最低的
\arrowitem 因为\fmahjong{1223p}也只有摸了\fmahjong{3p}才能形成两面
\arrowitem 因此切掉\fmahjong{2p}组成面子期待要\fmahjong{7s}周围的进张
\end{itemize}

\item 4连形vs中膨形

\textbf{例: 宝牌 \fmahjong{5z}} 

\mahjong{5667m3456p2378s55z}

\begin{itemize}
\arrowitem 万子跟饼子的两面比较如下 \newline
\begin{tabular}{|c|c|}
\hline 
万子部分的改良 & 饼子部分的改良 \\
\hline 
\fmahjong{45667m} & \fmahjong{23456p} \\
\hline 
\fmahjong{56677m} & \fmahjong{34456p} \\
\hline 
\fmahjong{55667m} & \fmahjong{34556p} \\
\hline 
\fmahjong{56678m} & \fmahjong{34567p} \\
\hline 
\end{tabular}

\arrowitem 由此可见中膨形不能形成3面,而4连形则有机会形成两面,比中膨形更好,因此切\fmahjong{6m}
\end{itemize}
\end{itemize}


\section{二度受}
所谓的二度受,也就是重复进张,说的是有两组搭子都需要同一张牌来帮自己组成面子,就比如以下这道例题


\textbf{例: 宝牌 \fmahjong{6p}} 

\mahjong{11178m3467p34567s}

\begin{itemize}
\arrowitem 这题就是典型的搭子超载了,\fmahjong{34567s}三面进张会看成两组搭子
\arrowitem 也就是说要在\fmahjong{34p}和\fmahjong{67p}选一组拆掉,因为他们都需要一个\fmahjong{5p}来帮自己组成面子
\arrowitem 那么答案就很明显了,\fmahjong{67p}因为宝牌是\fmahjong{6p},那肯定不能动(红亚树说过宝牌是恋人)
\arrowitem 另一方面如果切\fmahjong{78m}则损失掉\fmahjong{69m},而切\fmahjong{34p}的话损失的也只有\fmahjong{2p},毕竟\fmahjong{5p}能帮\fmahjong{67p}组成搭子
\arrowitem 所以只有拆\fmahjong{34p},要是场况不是很特殊的话,应该切掉离宝牌较近的\fmahjong{4p}
\end{itemize}

连对其实也算是二度受的一种

\textbf{例: 宝牌 \fmahjong{1m}} 

\mahjong{11178m3344p34567s}

\begin{itemize}
\arrowitem 同样的搭子的超载,但要是能摸到\fmahjong{25p}的话不是还能有个一杯口么?
\arrowitem 这手牌因为有个宝牌暗刻,牌效率上应该拆掉重复进张希望更快能达到听牌甚至胡牌
\arrowitem 因此拆掉连对要追求速度,通常都是从内到外切,因此\fmahjong{4p}
\arrowitem 但是场上出现过一枚\fmahjong{3p}的话,自己能摸进\fmahjong{3p}的暗刻比摸进\fmahjong{4p}要低,这时候就得先打\fmahjong{3p}了
\end{itemize}

二度受的部分还有一种方法就是把他变成雀头

\textbf{例: 宝牌 \fmahjong{5z}} 

\mahjong{233m33467p345678s}

\begin{itemize}
\arrowitem 这题是断平系缺雀头的一向听,那么就是\fmahjong{2m}跟\fmahjong{4p}的选择去固定雀头
\arrowitem 对比一下之后的进张 \newline
切\fmahjong{2m}后 进张只有 \fmahjong{3m2356p} 5种16枚 \newline
切\fmahjong{4p}后 进张只有 \fmahjong{134m358p} 6种20枚
\arrowitem 因此切\fmahjong{4p} 进张面会更广
\end{itemize}

\section{5 Block打法}

通常情况下和牌只需要4组面子+1组对子组成,那么这种状态就是5 Block状态。如果面子后补有5个的话那就成了6 Block。

\end{document}