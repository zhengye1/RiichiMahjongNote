\documentclass[小V的日麻笔记.tex]{subfiles}
\begin{document}
\chapter{手顺篇}
一副牌拿到手就应该决定好其中一条路线
\begin{itemize}
\item 断平系
\item 一色系
\item 门清宝牌系
\item 副露宝牌系
\item 副露过庄的
\item 无理系
\end{itemize}

每种路线都有他不同的手顺
\section{立直手顺}
1巡\ 东1局\ 南家\ 宝牌\fmahjong{9p}\\
\fmahjong{149m2399p556s135z}\ 摸\fmahjong{2s} \\
\begin{itemize}
\arrowitem 打\fmahjong{1m}
\arrowitem 最难组成面子的是只能成刻子的字牌,但!
\arrowitem 有4的筋牌1或者有6的筋牌9难以用上
\arrowitem 字牌暂时还能当个安留着
\end{itemize}

2巡\ 东1局\ 南家\ 宝牌\fmahjong{9p}\\
\fmahjong{49m2399p2556s135z}\ 摸\fmahjong{7m} \\
\begin{itemize}
\arrowitem 打\fmahjong{3z}
\arrowitem 234三色可能性,客风用途最小,加上目前面子不足
\end{itemize}

3巡\ 东1局\ 南家\ 宝牌\fmahjong{9p}\\
\fmahjong{479m2399p2556s15z}\ 摸\fmahjong{3s} \\
\begin{itemize}
\arrowitem 打\fmahjong{5z}
\arrowitem 同样搭子不足,但自己是宝牌2,把\fmahjong{4m}当面子后补
\arrowitem 打\fmahjong{1z}的话亲家碰了就送他2番了
\end{itemize}

4巡\ 东1局\ 南家\ 宝牌\fmahjong{9p}\\
\fmahjong{479m2399p23556s1z}\ 摸\fmahjong{8m} \\
\begin{itemize}
\arrowitem 打\fmahjong{4m}
\arrowitem 虽然说索子是二度受,但摸上\fmahjong{9p5s}成暗刻就是好形一向听
\arrowitem 还留着\fmahjong{1z}是一旦亲家碰了,这\fmahjong{4m5s}就有可能冒着风险切出来
\end{itemize}

5巡\ 东1局\ 南家\ 宝牌\fmahjong{9p}\\
\fmahjong{789m2399p23556s1z}\ 摸\fmahjong{4s} \\
\begin{itemize}
\arrowitem \fmahjong{1z},但实在想要平和的话就\fmahjong{5s},否则进张全要
\arrowitem 打\fmahjong{1z}的进张是\fmahjong{149p147s5s} 7种23枚
\arrowitem 打\fmahjong{5z}的进张是\fmahjong{14p147s} 5种19枚
\end{itemize}

后续没啥其他问题的话就立!

\section{平胡手顺}
1巡\ 东1局\ 南家\ 宝牌\fmahjong{9p}\\
\fmahjong{340m2379p556s135z}\ 摸\fmahjong{4s} \\
\begin{itemize}
\arrowitem 打\fmahjong{5z}
\arrowitem 4组面子缺雀头,平和明显,不能是役牌雀头先处理掉
\end{itemize}

通常字牌的处理顺序如下:
\begin{enumerate}
\item 自己的客风
\item 三元牌
\item 场风牌
\item 他家客风牌
\item 自己的连风牌
\end{enumerate}

2巡\ 东1局\ 南家\ 宝牌\fmahjong{9p}\\
\fmahjong{340m2379p4556s13z}\ 摸\fmahjong{9p} \\
\begin{itemize}
\arrowitem 打\fmahjong{3z}
\arrowitem 有速度有打点了
\arrowitem \fmahjong{4556s}拆成\fmahjong{45s}跟\fmahjong{56s}两个2面去考虑
\arrowitem 再说一旦摸了\fmahjong{68p}就可以只有1组\fmahjong{456s}的面子了
\end{itemize}

3巡\ 东1局\ 南家\ 宝牌\fmahjong{9p}\\
\fmahjong{340m23799p4556s1z}\ 摸\fmahjong{3p} \\
\begin{itemize}
\arrowitem 打\fmahjong{1z}
\arrowitem 但为了防守打\fmahjong{37p5s}的进张变窄
\end{itemize}

4巡\ 东1局\ 南家\ 宝牌\fmahjong{9p}\\
\fmahjong{340m233799p4556s}\ 摸\fmahjong{6m} \\
\begin{itemize}
\arrowitem 打\fmahjong{3p}
\arrowitem 万子4连形听牌的容易形成好形,摸\fmahjong{27m}就三面,摸\fmahjong{45m}也是两面
\arrowitem 虽说索子是中膨形,比4连形差点,但也可以成为好形。
\arrowitem 打\fmahjong{3p}是20种68牌的一向听,打\fmahjong{7p}就是19种63牌。
\end{itemize}

5巡\ 东1局\ 南家\ 宝牌\fmahjong{9p}\\
\fmahjong{3406m23799p4556s}\ 摸\fmahjong{6p} \\
\begin{itemize}
\arrowitem 打\fmahjong{5s}
\arrowitem 摸\fmahjong{14p}或者\fmahjong{58p} 就是平和宝牌3的听牌
\end{itemize}

6巡摸\fmahjong{7m}打\fmahjong{6p} 来到第7巡 \\
7巡\ 东1局\ 南家\ 宝牌\fmahjong{9p}\\
\fmahjong{34067m23799p456s}\ 摸\fmahjong{3z} \\
\begin{itemize}
\arrowitem 打\fmahjong{7p}
\arrowitem 搭子齐全,宝牌雀头,需要一个防守牌,加上7巡可能有人已经听了
\end{itemize}

\section{役牌手顺}
有时候留役牌是为了副露做到更快更大
1巡\ 东1局\ 南家\ 宝牌\fmahjong{8m}\\
\fmahjong{2688m25p1234s235z}\ 摸\fmahjong{5z} \\
\begin{itemize}
\arrowitem 打\fmahjong{3z}
\arrowitem 宝牌2的牌,能碰白的话可以加速,客风用不上
\arrowitem 况且搭子不齐,留孤张数牌做面子
\end{itemize}

但此时上家打\fmahjong{5z}呢?
2巡\ 东1局\ 南家\ 宝牌\fmahjong{8m}\\
\fmahjong{2688m25p1234s255z}\ 打出\fmahjong{5z} \\
\begin{itemize}
\arrowitem 无视
\arrowitem 搭子不齐,碰了防御力下降
\arrowitem 况且后续要是看到平和的话还能拆\fmahjong{5z}
\end{itemize}

但到了7巡
7巡\ 东1局\ 南家\ 宝牌\fmahjong{8m}\\
\fmahjong{24688m456p234s55z}\ 打出第二枚\fmahjong{5z} \\
\begin{itemize}
\arrowitem 碰打\fmahjong{2m}
\arrowitem 第二枚了,第7巡了,搭子齐了,加速
\arrowitem 想打\fmahjong{6m}骗\fmahjong{3m}摸\fmahjong{0m}咋办..因此留下摸\fmahjong{0m}的机会
\end{itemize}

再看个例题 \\
1巡\ 东1局\ 南家\ 宝牌\fmahjong{2z}\\
\fmahjong{35689m378p112s55z}\ 摸\fmahjong{8p} \\
\begin{itemize}
\arrowitem 打\fmahjong{3m}
\arrowitem 起手面子足够,比较\fmahjong{3m-3p}
\arrowitem 打\fmahjong{3m}因为有\fmahjong{56m}的搭子去兼顾\fmahjong{4m}
\end{itemize}
2巡\ 东1局\ 南家\ 宝牌\fmahjong{2z}\\
\fmahjong{5689m3788p112s55z}\ 摸\fmahjong{4p} \\
\begin{itemize}
\arrowitem 打\fmahjong{8m}
\arrowitem 役牌,\fmahjong{56m}两面,\fmahjong{788p}好形搭子,那就是看动\fmahjong{89m}还是\fmahjong{112s}
\arrowitem \fmahjong{112s}是边张双碰复合形,不动,再说拆\fmahjong{89m}不损\fmahjong{7m}是有\fmahjong{56m}搭子
\arrowitem 拆搭子一般从内到外,因为外的牌安全度稍稍高那么一点
\end{itemize}
3巡摸\fmahjong{5p}打\fmahjong{9m} \\
4巡摸\fmahjong{3s}打\fmahjong{1s} \\
5巡\ 东1局\ 南家\ 宝牌\fmahjong{2z}\\
\fmahjong{56m345788p123s55z}\ 摸\fmahjong{4m} \\
\begin{itemize}
\arrowitem 立\fmahjong{8p}
\arrowitem 立\fmahjong{8p}荣和1,300。 立\fmahjong{7p}荣和\fmahjong{8p}1,300,荣和\fmahjong{5z}才2,600。这高低目才提高1,300意义不大。
\arrowitem 但是手上有1枚宝牌的话就双碰了。原本是立宝牌1的2,600 胡到高目的\fmahjong{5z}就是5,200。
\arrowitem 宝牌2的话就两面,因为立直宝牌2加自摸就是满贯。选双碰胡了高目\fmahjong{5z}也是立宝牌2\fmahjong{5z}同样满贯,没区别了。
\end{itemize}
再看一道 \\
1巡\ 东1局\ 南家\ 宝牌\fmahjong{2z}\\
\fmahjong{357m127p46s55667z}\ 摸\fmahjong{6m} \\
\begin{itemize}
\arrowitem 打\fmahjong{7p}
\arrowitem 役牌2组,混一色,对对,全带都有可能但没宝牌..
\arrowitem 摸到\fmahjong{7z}还能看看大三元,总之能碰就碰
\arrowitem 比较\fmahjong{3m-7p}的话,\fmahjong{3567m}摸个\fmahjong{4m}就是三面听了,比留\fmahjong{7p}强
\end{itemize}

\section{混一色手顺}
1巡\ 东1局\ 南家\ 宝牌\fmahjong{2z}\\
\fmahjong{146688m18s55667z}\ 摸\fmahjong{4z} \\
\begin{itemize}
\arrowitem 打\fmahjong{1s}
\arrowitem 染手太明显了
\arrowitem 先打\fmahjong{8s}也行,但先打\fmahjong{1s}不会引人注意
\end{itemize}

3巡\ 东1局\ 南家\ 宝牌\fmahjong{2z}\\
\fmahjong{1346688m455667z}\ 摸\fmahjong{6m} \\
\begin{itemize}
\arrowitem 打\fmahjong{1m}
\arrowitem \fmahjong{1m4z}留下安全度更高的\fmahjong{4z}
\end{itemize}

另一个手牌上的局面:\\
4巡\ 东1局\ 南家\ 宝牌\fmahjong{2z}\\
\fmahjong{78p220666788s55z}\ 摸\fmahjong{5z} \\
\begin{itemize}
\arrowitem 打\fmahjong{8s}
\arrowitem 细分搭子会发现不要染手更好
\arrowitem \fmahjong{78p}/\fmahjong{22s}/\fmahjong{555z}/\fmahjong{0666788s}
\arrowitem \fmahjong{0666788s}既有\fmahjong{47s}也有\fmahjong{69s}进张。\fmahjong{2s}成暗刻也是1面子1雀头超好形搭子。
\arrowitem 按照染手来分block的话 \\
\fmahjong{22s}/\fmahjong{555z}/\fmahjong{678s}/\fmahjong{06s}/\fmahjong{68s}
\arrowitem 没算\fmahjong{78p}是因为不算搭子,虽然说\fmahjong{0666788s}有坎\fmahjong{7s}跟\fmahjong{47s}进张,但这\fmahjong{7s}就成了二度受,速度会慢
\end{itemize}

已经听牌的情况呢?\\
8巡\ 东1局\ 南家\ 宝牌\fmahjong{2z}\\
\fmahjong{779p22067s-55'5z}\  摸\fmahjong{8s} \\
\begin{itemize}
\arrowitem 打\fmahjong{7p}
\arrowitem 先听着,要是碰上\fmahjong{2s}就单骑\fmahjong{9p}再看看能不能过渡到索子染手
\end{itemize}
\section{断么手顺}
\section{变则手手顺}
\end{document}