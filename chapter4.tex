\documentclass[小V的日麻笔记.tex]{subfiles}
\begin{document}
\chapter{防守篇}
麻将是4人游戏,但和牌只有1人。所以不能进攻的时候就好好防守。

\section{了解听牌速度}
\fbox{
\begin{minipage}{30em}
$\star\star\star$ 3面听以上(含3面单骑,但不包括对碰的3面)\\ 
例子:\\
\mahjong{23456m111p111s55z} 听\fmahjong{147m} \\
\mahjong{234567m2333p111s} 听\fmahjong{124p} 变则3面 \\
$\star\star$ 2面/含对碰的变则3面/带字牌幺九对或者单骑 \\ 
例子:\\
\mahjong{123456m33345p11s} 听\fmahjong{36p1s} 带双碰的变则3面, 总听牌数7枚比两面少1枚 \\
$\star$ 坎张/边张/对碰/单骑
\end{minipage}}

\section{了解一向听时的速度}
\fbox{
\begin{minipage}{30em}
$\star\star$ 不管进什么听牌都能形成3面听以上\\ 
$\star$ 不管进什么听牌都能形成2面听以上 \\ 
0速 - 有可能成为坎/边/单骑等速度为1的听牌
\end{minipage}}

\section{巡目平衡}
手牌两向听的话当做速度0 \\
\fbox{
\begin{minipage}{30em}
1-9巡 \\ 速度0就朝和牌发展\\ 
10-14巡\\ 还是速度0就得开始考虑防守 \\ 
15巡之后\\ 速度1的话就考虑防守打牌
\end{minipage}}

\section{大牌基础以及不要在意罚符}
手牌若能做到3番以上,把速度提升1级。但不要为了强行不给罚符去对攻

\section{猛烈进攻下}
当然全跑是不行的..可以根据以下来判断 \\
\fbox{
\begin{minipage}{30em}
\begin{itemize}
\item 亲家立直时候 \\
$\star\star$ 速度是2的话就对抗,否则守备重视
\item 闲家立直的时候 \\
$\star$ 速度1就对抗,否则守备重视
\item 2副露的时候 \\
$\star$ 速度1就对抗,否则守备重视
\item 其他攻击 无视
\end{itemize}
\end{minipage}
}

\section{牌的安全度}
要完美防守就得知道牌的安全度,简单来说就是 \\
\textbf{危险} 
【会点两面听牌的】 → 【不太可能点两面但会点双碰坎张的】 → 【只会点单骑的】 \textbf{安全}
\\
把这个细分就有以下几种:
\begin{itemize}
\item 100\% 不会点的 \\
立直家现物 → 场上出现的3枚字牌 → 场上出现的2枚字牌 = 场上出现的2枚筋牌1或者9 = 场上出现的2枚NC→自己手上的3枚字牌
\item 非常难点的 \\
筋牌19 = 立直后切过的筋牌(1或者9)→ NC
\item 比较难点的 \\
立直后切的筋牌(2~8)→立直前切的筋牌(2~8)→ 切过1枚的字牌
\item 可能会点但没安的时候 \\
生张字牌 → 无筋19牌 → OC
\item 普通危险牌 \\
立直宣言牌的筋(2~8) → 普通无筋
\end{itemize}

注意: 立直宣言牌的筋不是那么安全的!

\section{弃胡的切牌顺}
\begin{enumerate}
\item 决定弃胡的话争取手里是全员的安全牌 \\
一般手顺是过立直家的牌 → 全员都能过的牌
\item 应该弃胡的牌就不要想着不拆牌 \\
直接斩断羁绊
\item 关注切立直家现物以外的牌的人 \\
红色信号:切的是无筋 \\
黄色信号:筋牌,OC,生张字牌这种看起来挺危险的
\item 最优先的还是立直家的安牌
\end{enumerate}

\section{读染手}
牌河简单例子:\\
\mahjong{4m5p2z8m8m3z}\\
\mahjong{65z}\\
这种舍牌索子染手明显,不要觉得这张索子是OC就能打
\\
另一种情况:\\
\mahjong{236z4m5z5p}\\
\mahjong{88m} \\
这\fmahjong{88m}太怪了,大概率是染手听了。
\\
因此染手很容易读

\section{读对对}
即便是碰过一次,如果对手有“确定双碰的切牌”,当他对对。比如牌河是这种的碰了\fmahjong{4p}, 大概率对对\\ 
\mahjong{4z1m8m3p5z}
\\
切\fmahjong{3p}打\fmahjong{5z}表示\fmahjong{345p5z}不需要\fmahjong{25p}进张,那就是重视对子无视顺子

\section{读区域}
如果某个区域空了个洞,筋跟壁就难通过。比如\\
\mahjong{43z15s29m}\\
\mahjong{3p6z4p}
\\
虽说\fmahjong{5s}有筋牌\fmahjong{28s},\fmahjong{4p}有筋牌\fmahjong{17p},但整体\fmahjong{2s1p}安全,\fmahjong{8s7p}有点不可靠
\\
因此场况差的区域筋牌很危险

\section{手里的壁更优先}
因为牌河出现的壁,有可能有人蹦着这个去

\section{立直宣言牌的筋最危险}
俗称“宣言牌引挂立直”,不要被骗啦~

\section{用安全牌组成搭子}
简单例子:
自己北家\\
东家弃牌 \fmahjong{34z5m5s}\\
南家弃牌 \fmahjong{35z4s3p}\\
西家弃牌 \fmahjong{1m6z3s5p}\\
自身手牌 \fmahjong{13589m33534488s-7p} 宝牌\fmahjong{3m} \\
能打的牌只有\fmahjong{7p}和\fmahjong{89m},先拆\fmahjong{89m}攻防兼备

\section{不要透露信息}

\end{document}