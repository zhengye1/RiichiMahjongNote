\documentclass[UTF8,12pt,titlepage,oneside]{ctexbook}
\usepackage[height=1.5\baselineskip]{mahjong}
\usepackage{fancyhdr}
\usepackage{subfigure}
\usepackage[subfigure]{tocloft}
\usepackage{bookmark}
\usepackage{pifont}
\hypersetup{hidelinks,
colorlinks=true,
allcolors=black,
pdfstartview=Fit,
breaklinks=true
}
\renewcommand{\cftchapleader}{\cftdotfill{\cftdotsep}}
\renewcommand{\baselinestretch}{1.5} \normalsize
\newcommand{\fmahjong}[1]{\raisebox{-3mm}{\mahjong{#1}}}


\title{小V的日麻笔记}
\author{小V}
\date{\today}
\setlength{\headheight}{16pt}
\begin{document}
\maketitle
\pagestyle{empty}
\tableofcontents

\pagestyle{fancy}
\chapter{心态篇}
\section{要了解自己的成绩}
如果可能的话,以1000个半庄为一个单位,去检阅自己的成绩。如果1000个半庄太过困难的话,可以以100个半庄为单位来检测, 因为只有了解到自己的成绩才知道自己的麻将水平。平均顺位在2.3到2.4之间的话可以称之为“胜者组”。反之则是“败者组”。所以一旦进入了败者组就说明自己的麻将技术是有问题的,需要去改进。

一旦突破平均顺位是2.4以下的话,无疑自己就可称之为是个“麻将强者”啦。

\section{麻将即使痛苦也要喜欢}
麻将是个4人游戏,有时候会对对手有抱怨的情绪。不过这些情绪是成为麻将强者的必经之路。有时候牌局可能很痛苦,但也要坦然接受,去面对。所以无论何时都要愉快的去打牌。


\cleardoublepage
\chapter{做牌}
\section{基本手筋与手顺}
二阶堂亚树在红亚树说过,无论什么巡目,随意的切牌都不会有好下场。因此在做牌方面要学会从牌效上打牌。
那么\textbf{牌效}是什么?红亚树上定义的牌效是,打什么牌进张面最广。在鱼谷的书当中,定义的是平衡距离听牌的速度与和牌的速度,来做出最佳的选择。无论是哪种定义方式也离不开最基本的手筋与手顺。

那么什么是手筋,什么是手顺?
手筋,指的是基于战略做出的打牌方针
手顺,则是顾名思义切牌的顺序

所以来学习基础的手筋把。

一般一副手牌的切牌顺序如下:
\begin{itemize}
\item 在面子足够的情况下切有筋的端牌(既有4的时候切浮牌1,有6的时候切浮牌的9)

\textbf{例: 宝牌 \fmahjong{5z}}

\mahjong{14m1799p23467s554z}
\begin{itemize}
\item[\ding{213}] 切 \fmahjong{1m}
\item[\ding{213}] 能够组成面子的已经有\fmahjong{799p-234s-67s-55z}这4组
\item[\ding{213}] 即使切\fmahjong{4z},摸到了\fmahjong{23m}还是要切掉\fmahjong{1m}
\end{itemize}

\item 字牌
\begin{itemize}
\item 首先要考虑的是有无平和?因为平和需要非役牌的雀头

\textbf{例: 东1局\ 西家\ 第1巡\ 宝牌 \fmahjong{7p}}

\mahjong{23678m36789p34s25z}

\begin{itemize}
\item[\ding{213}] 这手牌能看出是个平和的,优先处理掉会减平和那一翻的\fmahjong{5z}会更佳
\end{itemize}

\item 然后就是最没价值的字牌

\textbf{例: 东1局\ 西家\ 第1巡\ 宝牌 \fmahjong{7p}}

\mahjong{13668m35799p3s125z}
\begin{itemize}
\item[\ding{213}] 多张字牌的时候,如果没有手役,一般都是客风→役牌→双风牌
\item[\ding{213}] 因此这手牌价值最小的就是客风的 \fmahjong{2z}
\item[\ding{213}] \fmahjong{1z}留最后是因为对自己来说只是单纯的役牌,然而优先打出去被庄家碰出有了确定的2翻的话是不利的,应该等人打出之后跟着打
\end{itemize}
\end{itemize}

\item 端牌

处理好字牌,筋上的端牌之后就是浮牌的端牌了

\textbf{例: 宝牌 \raisebox{-3mm}{\mahjong{5z}}}

\mahjong{1668m3799p2788s55z}

\begin{itemize}
\item[\ding{213}] 这一手牌只有1张价值最低的端牌,那就是 \raisebox{-3mm}{\mahjong{1m}}
\item[\ding{213}] 牌效率上讲数牌越往中间靠就越好。
\end{itemize}

但端牌之间也有一个比较,就是看看有没有相差4的的牌。

\textbf{例: 宝牌 \fmahjong{5z}}

\mahjong{15m167p234779s555z}

\begin{itemize}
\item[\ding{213}] 这手牌有两张浮牌,分别是 \fmahjong{1m} 和 \fmahjong{1p}
\item[\ding{213}] 但最佳留下的应该是 \fmahjong{1m}, 因为跟 \fmahjong{5m} 相差了4,换句话说一旦摸进了 \fmahjong{3m} 就能形成 \fmahjong{135m} 的两坎形式
\item[\ding{213}] 因此这题应该切 \fmahjong{1p}
\end{itemize}

\item 但是也有一些特殊情况,比如跟一色手,七对子,全带,对对等有手役的时候

\textbf{例: 宝牌 \fmahjong{7p}}

\mahjong{113467889m68s225z}

\begin{itemize}
\item[\ding{213}] 这手牌很明显的万子染手,这种情况下就应该从最中间也是较为危险的\fmahjong{6s}开始
\end{itemize}

\textbf{例: 宝牌 \fmahjong{9m}} 

\mahjong{14688m15p12s3345z-9m}

\begin{itemize}
\item[\ding{213}] 这一道题来自红亚树,选择的第一打是 \fmahjong{5p}
\item[\ding{213}] 手牌一般,看看能不能往混一色发展,需要注意的是此时他家打出\fmahjong{3z}的时候是不能碰的,至少要摸成了宝牌对子或者役牌对子才能决定混一色路线。
\item[\ding{213}] 况且这牌还有可能发展成一气通贯,总之副露的混一色本身也只有2翻,打点很低,至少要有额外的1翻或者2翻达到满贯的目的就最好了。
\end{itemize}

\end{itemize}

\cleardoublepage
\chapter{进攻篇}
进攻方面
\cleardoublepage
\chapter{防守篇}
防守方面
\cleardoublepage
\end{document}
