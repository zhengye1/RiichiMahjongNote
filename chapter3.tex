\documentclass[小V的日麻笔记.tex]{subfiles}
\begin{document}
\chapter{进攻篇}
\section{立直篇}

进行强力进攻就得有以下手段:
\begin{enumerate}
\item 做牌比人大
\item 和牌比人快
\end{enumerate}

立直可以算是可以把他做大的方法之一,先看优缺点

\textbf{\emph{优点}}
\begin{enumerate}
\item 起和役
\item 打点提升
\item 对手没法直线做牌
\end{enumerate}

\textbf{\emph{缺点}}
\begin{enumerate}
\item 只能摸切
\item 要和的牌不那么容易被打出来
\end{enumerate}

\textbf{\emph{严禁等待一巡摸切立!}}
\\
要说原因的话,在一向听的时候就得开始考虑\textbf{[如果听牌了要不要立直]},而不是等到听牌了再去想
\\
\\
那么立直的判断基准呢?根据下列5个在细分
\begin{description}
\arrowitem 两面听牌
\arrowitem 坎张听牌
\arrowitem 跟赤有关的听牌
\arrowitem 双碰听牌
\arrowitem 单骑听牌
\end{description}

\section{立直判断-两面听牌情况}
\begin{itemize}
\item 先手两面立直
\\
先手两面9成直接立了, 重点是\textbf{先手}!
\\
\item 默听是满贯以上
\\
根据局面来判断,立直也不能说错
\\
\item 第三排(13巡)之后
\\
很大机会已经是后手,建议是默听,但如果荣和没役的话而且两面听牌立
\\
\item 1位或者很贴近1位
\\
默听就好,但是默听没役的还是立
\end{itemize}


\section{立直判断-坎张情况}
满足以下5点的其中3点,那就立把。实际上即使满足3点,欠缺1或者2的话也有可能是默听的
\begin{enumerate}
\item 改良枚数是胡牌枚数的2倍或以下
\\
改良枚数,指能把牌改成好形或者让牌的打点上升的枚数。比如原本是坎张变两面听牌,原本也只有2000点变成了3900这种
\item 立直后有2600以上的打点
\item 荣和的牌非常容易打出来
\item 先手
\item 亲家
\end{enumerate}

\textbf{例1 宝牌\fmahjong{5s}}
\\
\mahjong{22m13456s234678p}
\\
目前听牌只有坎\fmahjong{2s} 4枚,然而改良有\fmahjong{2m 3s 4s 5s 6s 7s 8s} 一共7种22枚, 改良枚数是听牌枚数2倍以上。 (要注意的是\fmahjong{1s}不算改良,因为既不能提高打点也不能改成好形)。因此这种牌达到满足3个条件也会默听,或者切\fmahjong{1s} 拒听
\\
\hrulefill 
\\
\textbf{例2 东1 南家 5巡 宝牌\fmahjong{4z}}
\\
\mahjong{24789m05p123456s}
\\
加上立直打点暂时2600点以上,改良也只有1种\fmahjong{5m}, 先手的的话就立了
\\
\hrulefill 
\\
\textbf{例3 东1 南家 5巡 宝牌\fmahjong{4z}}
\\
\mahjong{12379m05678p789s}
\\
这一手的改良有\fmahjong{6m}和\fmahjong{9p},2种8枚,刚好待牌\fmahjong{8m}2倍(毕竟1种4枚),就不用等了。而且打点也有2600以上,因此不等改良直接立。
\\
\hrulefill 
\\
\textbf{例4 东1 南家 5巡 宝牌\fmahjong{4z}}
\\
\mahjong{13456m05678p789s}
\\
这一手的改良有\fmahjong{4m 5m 7m 5p},4种12枚,刚好待牌\fmahjong{2m}2倍以上(毕竟1种4枚),条件1不满足。虽然打点也有2600以上,但还是不推荐立直
\\
\hrulefill 
\\
\textbf{例5 东1 南家 5巡 宝牌\fmahjong{4z}}
\\
\mahjong{13456m33678p789s}
\\
这一手的改良有\fmahjong{4m 5m 7m 5p},4种12枚,刚好待牌\fmahjong{2m}2倍以上(毕竟1种4枚),条件1不满足。打点也只有1300,条件2也不满足,默听等自摸

\section{立直判断-跟赤有关的听牌}
带赤的坎张立!
\\
\textbf{例1 东1 北家 5巡 宝牌\fmahjong{4z}}
\\
\mahjong{30789m11234s234p}
\\
虽然看到了有三色可能,但摸了\fmahjong{2m}也不是三色确,还要打掉一个\fmahjong{0m},打点也降了,虽说改良是\fmahjong{2m 6m}但会导致打点下降,所以还是立了
\\
\hrulefill 
\\
\textbf{例2 东1 北家 5巡 宝牌\fmahjong{4z}}
\\
\mahjong{07m11789s123p222z}
\\
这一手摸到了\fmahjong{89m}有全带,摸\fmahjong{4m}是改良。但摸\fmahjong{8m}也不能全带确,摸到\fmahjong{9m}的话还是坎张听牌,还是立了。
\\

除非改良能有确定的手役,不然要浪费一个宝牌不值得

\section{立直判断-双碰听牌}
\begin{itemize}
\arrowitem 如果待牌好,就无视打点立,比如字牌啊,幺九牌啊,勉强的2-8牌
\arrowitem 有役但打点低,待牌还差,就默听 
\arrowitem 待牌差的话就看打点跟改良的情况
\end{itemize}

\section{立直判断-单骑听牌}
即使是单骑听牌,也要选一个待牌好的
\\
\textbf{例1 东1 南家 5巡 宝牌\fmahjong{8s}}
\\
\mahjong{345678m7p678s333z}
\\
虽然已经听牌了,但单骑一个没啥信息中张牌算了,默听等万子多面听啊,678三色平和啊或者更好被打出来的牌再立直
\\
\hrulefill 
\\
\textbf{例1 东1 南家 5巡 宝牌\fmahjong{8s}}
\\
\mahjong{345678m234p678s3z}
\\
这种的牌就默听等一下好了,摸到断么也有2600了。要是能变成门断平加立直中个一发或者里就能轻易满贯

\section{平和立直判断}
结论上讲,子家的话平和nomi是不建议立直的。能提升的打点太低了。 最好是有dora 1或者dora 2的情况立,但亲家的话立! 

\section{偏听}
树老师麻将格言第14条
\\
无宝牌的低目无役的偏听,差距是2番以上时的默听

\section{宝牌是恋人}
虽然说宝牌是恋人,但有时候也有不得不打的时候,比如手役确定的时候

\textbf{例1 宝牌\fmahjong{1p}}
\\
\mahjong{234m123499p23478s}
\\
三色确定,打宝牌立直!
\\
\hrulefill
\textbf{例2 宝牌\fmahjong{5m}}
\\
\mahjong{2233445m23499p78s}
\\
宝牌跟一杯口,应该打\fmahjong{2m}放弃一杯口留下宝牌
\\
如果手役跟宝牌点数同等,优先留宝牌。
\\
\hrulefill
\textbf{例3 宝牌\fmahjong{2m}}
\\
\mahjong{2340m23499p12378s}
\\
这一手打\fmahjong{0m}。即使被碰也是加多1番,反之一旦他家碰了\fmahjong{2m}那就是7700起步了。

\section{手役稳定性}
重视宝牌也不能忽视手役重要性。
\\
优先级: \textbf{速度 \textgreater 宝牌 \textgreater 手役}
\\
\textbf{例1 宝牌\fmahjong{5z}}
\\
\mahjong{23456789m333p34s4p}
\\
有一气可能但也有三色可能,都是2番,从听牌形看优先三色
\\
\mahjong{23456789m33p345s}
\\
\mahjong{234567m23334p34s}
\\
荣和高目都是满贯,但低目的话第一个就只有立平,第二个有立平断么,而且自摸高目还是跳满。
\\
\hrulefill
\textbf{例2 宝牌\fmahjong{5z}}
\\
\mahjong{3456789m3345p45s6p}
\\
如果打\fmahjong{3m}就是高目的456三色,但没断么;但打\fmahjong{9m}就是断么确但没三色,因此比起不确定的三色,确定的断么能稳定打点。
\\
\hrulefill
\textbf{例3 宝牌\fmahjong{5z}}
\\
\mahjong{1234456s2345678p}
\\
打\fmahjong{258p}听\fmahjong{147s},\fmahjong{47s}是平和高目;打\fmahjong{1s}可以听断么确的\fmahjong{258p}。因此优先考虑断么,加上还能副露。

\section{副露}
需要副露的情况那就是牌快,牌大。牌大的基准话一发有里规则或者有赤的3900左右,牌快的基准能8巡内听牌的。 但有时候不能同时满足,那么牌快但小或者牌慢但大怎么办呢
\\
牌快但小的场合
\\
\textbf{例1 东1 南家 5巡 宝牌\fmahjong{2z}}
\\
\mahjong{13456m33s6788p11z}
\\
第一张\fmahjong{1z}直接碰。因为门清的话摸了\fmahjong{2m} 成了1枚现的\fmahjong{1z}跟\fmahjong{3s}的听牌。虽说可以做梦立直一发摸\fmahjong{1z}满贯,但毕竟是梦啊!!即使门清摸了\fmahjong{1z}暗刻,是个坎张听牌,也是很难立直默听也只有1300,跟碰听和1000差不多。
\\
再说碰了\fmahjong{1z}也可以积极的等待摸或者吃\fmahjong{457m}形成好形听牌
\\
但不是所有一次副露了就全部都要副露到底
\\
\textbf{例2 南2 西家 7巡 宝牌\fmahjong{8p} 38,000点1位}
\\
\mahjong{23m34s3566p44z-22'2z} 
\\
没有宝牌,为了过局碰了个\fmahjong{2z},场上出现第二张\fmahjong{4z},直接无视,因为会没了防御力。碰了也是2副露1向听,对手一旦反击会导致没安。这种手就等吃碰了两面的或者吃了\fmahjong{4p}或者碰了\fmahjong{6p}之后再打\fmahjong{4z}去斗快
\\
\textbf{例3 南2 西家 7巡 宝牌\fmahjong{2z} 28,000 的3位}
\\
\mahjong{12m34567s357p-22'2z}
\\
碰了役牌\fmahjong{2z},上家打了\fmahjong{5s},不吃。
\\
虽然吃了进入1向听,但得用个好形去吃。树老师格言23条, 用两面副露是雀士之耻。而且三面张比较容易摸的到。
\\
要记住副露是为了解决掉愚形而争取做成好形的牌
\\
但如果已经副露的牌再次副露能从1向听到听牌的话,就副露把。
\\
\textbf{例4 东2 南家 8巡 宝牌\fmahjong{8p}}
\\
\mahjong{33067m89p23s3z-11'1z}
\\
上家打出\fmahjong{14s}的话直接吃,虽然是坎\fmahjong{7p}听牌但至少也是能胡了。
\\
\hrulefill
\textbf{例5 南家 8巡 宝牌\fmahjong{8p}}
\\
\mahjong{111366789p2z-11'1z}
\\
上家打出\fmahjong{2p}的话直接吃,虽然是\fmahjong{16p}双碰但留着1向听跟满贯听牌还是有巨大差别的。再说吃了只有也有好形的改良

\section{役牌的副露判断}
分3个case:
\begin{itemize}
\item 绝对能碰的
\item 积极碰也是OK的
\item 最后别碰的
\end{itemize}

\textbf{绝对能碰的情况}
\\
最好有以下几种情况
\begin{enumerate}
\item 碰听
\item 宝牌2张以上
\item 碰之后是满贯
\end{enumerate}

1的情况基本都是碰的,但也有例外
\\
\textbf{例1 东1 南家 6巡 宝牌\fmahjong{5p}}
\\
\mahjong{3455m3p334455s66z}
\\
一杯口啊345三色满贯可能性最后别浪费
\\
\hrulefill
\\
\textbf{例2 东1 南家 6巡 宝牌\fmahjong{9m}}
\\
\mahjong{24699m47p1237s55z}
\\
形状虽然不好,但有\fmahjong{5z}还是先碰了把,毕竟手上2张宝牌\mahjong{9m}容易做成满贯 
\\
\hrulefill
\\
\textbf{例3 东1 南家 6巡 宝牌\fmahjong{4z}}
\\
\mahjong{5p1134577s45567z}
\\
有\fmahjong{5z}的话先碰,然后往索子染手靠,最坏也有3900打点,完全OK
\\
换句话说有明确有役,能保证最低3900打点也能看到满贯的,可以积极的碰。
\\
\textbf{积极副露都OK的情况}
\\
有以下这2种的话OK
\begin{enumerate}
\item 碰了是1向听
\item 碰了是好形2向听
\end{enumerate}

\textbf{例4 东1 南家 6巡 \fmahjong{5z}}
\\
\mahjong{1379m067p1178s66z}
\\
有愚型,但碰了\fmahjong{6z}是1向听。完全可以副露去加速和牌
\\
\hrulefill
\\
\textbf{例5 东1 南家 6巡 \fmahjong{5z}}
\\
\mahjong{2378m06p113s3466z}
\\
目前2向听,但有3个两面的2向听。副露了的话也比较容易和牌
\\
\textbf{不建议副露的情况}
\\
碰到这种的就不建议了
\begin{enumerate}
\item 碰了成愚形2向听
\item 后手
\end{enumerate}

\textbf{东1 南家 6巡 宝牌\fmahjong{2m}}
\\
\mahjong{579m1258p235s5577z}
\\
即便出了第二张\fmahjong{5z}也不建议碰。碰了没防御力而且最终形待牌还差
\\
\hrulefill
\\
\textbf{例5 东1 南家 10巡 \fmahjong{5z}}
\\
\mahjong{2378m06p113s3466z}
\\
跟前面例子同样手牌,但不同的是巡目是10巡。剩余枚数不一样,巡目也晚,为了好形还得冒险打\fmahjong{3m-3s-7s}。再说假如巡目早的话,有人立直了或者有人碰了宝牌啊,或者看起来有人染手啊,那还是不副露了。
\end{document}
